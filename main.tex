\documentclass{article}

% Packages
\usepackage[margin=1.5cm, includefoot, footskip=30pt]{geometry}
\usepackage{hyperref}
\usepackage{multicol}
\usepackage{booktabs}
\usepackage{xstring}
\usepackage{subcaption}
\usepackage{graphicx}
\usepackage{amsmath}
\usepackage{standalone}
\usepackage{fancyvrb}

\usepackage{tikz}
\usetikzlibrary{calc, shapes, patterns}

\usepackage[backend=biber, firstinits=false, backref=false, url=true,
            isbn=true, style=numeric]{biblatex}
\addbibresource{bibliography.bib}

% Title
\title{Reinforcement Learning Produces Dominant Strategies for the
Iterated Prisoner's Dilemma}
\author{Marc Harper \and Vincent Knight} % TODO Authors?
\date{}


\begin{document}

\maketitle

Key points:
* We trained several different strategies and they all are tournament
winners
* works for standard, noisy, and moran
* Complex strategies can fare well
* the best strategies are cooperative yet punitive
* copious references to W. and D. Ashlock

Todo:
* better title


\begin{abstract}
    We present several powerful strategies for the Iterated
    Prisoner's Dilemma created using reinforcement learning techniques
    (evolutionary and particle swarm algorithms). These strategies are
    trained to perform well against a corpus of over 100 distinct
    opponents, including many well-known strategies from the literature, and all
    the trained strategies win standard tournaments against the total collection
    of other opponents. We also trained variants to win noisy tournaments.
\end{abstract} 

\section{Introduction}\label{sec:introduction}

* Update on Axelrod Library
Description of strategy space and tournament capabilities
Includes many prior tournament winners (OmegaTFT, ZDGTFT2)
Some original strategies but mostly from the literature

% TODO: cite earlier tournaments

\section{The Strategy Archetypes}

The Axelrod library now contains many strategies based on machine learning
methods. Most are deterministic, use many rounds of memory, and perform
extremely well in tournaments.

\subsection{LookerUp}

The first strategy trained with reinforcement learning in the library is based
on lookup tables. The strategy encodes a set of deterministic responses
based on the opponent's first $n_1$ moves, the opponent's last $m_1$ moves, and
the players last $m_2$ moves. If $n_1 > 0$ then the player has infinite memory
depth, otherwise it has depth $\max{m_1, m_2}$. Although we tried various
combinations of $n_1, m_1,$ and $m_2$, the best performance at the time of
training was obtained for $n_1 = m_1 = m_2 = 2$ and generally for $n_1 > 0$. 
First impressions matter in the IPD. The library includes a strategy
called EvolvedLookerUp2_2_2 which is among the top strategies in the library.

This archetype can be used to train memory-$n$ strategies with the parameters
$n_1=0$ and $m_1=m_2=n$. For $n=1$, the result is WSLS.
% TODO: doublecheck that it is WSLS 

\subsection{PSO Gambler}

PSO Gambler is a stochastic variant of LookerUp. Instead of deterministically
encoded moves the lookup table emits probabilities which are
used to choose cooperation or defection. The library includes a player trained
with $n_1 = m_1 = m_2 = 2$ that is \emph{mostly deterministic}, with most of the
probabilities being 0 or 1. At one time this strategy outperformed
EvolvedLookerUp2_2_2.

This strategy type can be used to train arbitrary memory-$n$ strategies and a
memory one strategy was trained and is called PSO Gambler Mem 1, with
probabilities $\ldots$. Though it performs well in deterministic tournaments
it is not as good as the longer memory strategies, and is bested by a similar
strategy that also uses the first round of play.

% TODO: quantify how stochastic
% TODO: include memory one four vector

\subsection{ANN}

Strategies based on artificial neural networks can also be trained with an
evolutionary algorithm. A variety of features are computed from the history
of play such as the opponents trailing moves, the total number of cooperations
of the player and the opponent, and several others, which are then input
into a feed forward neural network with one layer and user-supplied width.
An inner layer with just five nodes performs quite well in both deterministic and
noisy tournaments. The output of the ANN used in this work is deterministic and
a stochastic variant that outputs probabilities rather than exact moves could
be easily created.

% TODO: Cite earlier ANN papers including Ashlock
% TODO: list features

\subsection{Finite State Machines}

We used strategies based on finite state machines to create a number of
strategies. These strategies are deterministic and are efficient computational.
At each state the machine transitions to a new state and plays a specified move
depending on the opponent's last action. These strategies can encode a variety
of other strategies, including classic strategies like TitForTat, encode
handshakes, and gruding strategies that always defect after an opponent
defection.

% TODO: Cite Ashlock FSM papers
% TODO: One or more diagrams

\subsection{Hidden Markov Models}

We also trained stochastic versions of finite state machine players called
hidden Markov model players or HMMs. These strategies also encode an internal
state with probabilistic transitions to other states and cooperate or defect
with various probabilities at each state. These are the best performing
stochastic strategies in the library but take longer to train due to their
stochasticity.

\subsection{Meta Strategies}

Last but not least there are several strategies based on ensemble methods that
are common in machine learning called Meta strategies. These strategies are
composed of a team of other strategies and each is polled for its desired next
move. The ensemble then selects the next move based on some rule, such as the
consensus vote in the case of MetaMajority or the best individual performance
in the case of MetaWinner. These strategies were among the best in the library
before the inclusion of those trained by reinforcement learning.

\section{Results}

\subsection{Standard Tournament}

We conducted a tournament with all strategies in the Axelrod library. The top
11 performing strategies by mean payoff are all strategies trained to maximize
total payoff against a subset of the strategies (Figure \ref{}). The next strategy is 
Desired Belief Strategy (DBS) which actively analyzes the opponent and responds
accordingly. The next two strategies are Winner12, based on a lookup table, and
Fool Me Once, a strategy that defects indefinitely on the second defection.

Pairwise payoff results are given as a heatmap (Figure \ref{}) which
shows that the many strategies achieve mutual cooperation with each other and
that the top performing strategies never defect first but are able to exploit
weaker strategies that attempt to defect.

The strategies that win the most matches are Defector, Aggrevater, and then a
series of ensemble strategies called MetaWinner, that combine the strategies
of many other strategies and take a consensus vote for each move. These are
followed by handshaking and zero determinant strategies.

% Table of best strategies
% \includegraphics{strategies_std_boxplot.svg}
% \includegraphics{strategies_std_payoff.svg}
% \includegraphics{strategies_std_winplot.svg}

\subsection{Noisy Tournament}

We also ran noisy tournaments in which there is a 5\% chance that an action
is flipped. The best performing strategies in mean payoff are DBS, designed
to correct for noise, followed by two finite state machine strategies trained
in the presence of noise, and OmegaTFT (also designed to handle noise). The
strategies trained in the presence of noise are also among the best performers
in the absence of noise.

The cluster of mutually cooperative strategies is broken by the noise at 5\%. A
similar collection of players excels at winning matches but again they have
a poor total payoff.

% Table of best strategies
% \includegraphics{strategies_noisy_boxplot.svg}
% \includegraphics{strategies_noisy_payoff.svg}
% \includegraphics{strategies_noisy_winplot.svg}

\section{Methods}

We trained a variety of strategies using evolutionary algorithms, and in the
case of PSO Gambler particle swarm algorithms. The evolutionary algorithms
used standard techniques, varying strategies by mutation and crossover, and
evaluating the performance against each opponent for many repetitions. The
best performing strategies in each generation are persisted, variants created,
and objective functions computed again. This process continues for approxaimtely
200 generations or until strategies no longer improve significantly.

% Diagram with ANN training performance

All training code is available on github. The are objective functions for
* total payoff
* total payoff difference
* total Moran process wins (fixation probability)

New strategies can be easily trained with variations including noise, spatial
structure, and probabilistically ending matches.

\section{Discussion}

The tournament results indicate that pretrained strategies are generally better
than human designed strategies at maximizing payoff against a diverse set of
opponents. A simple evolutionary algorithm produces strategies based on multiple
standard machine learning techniques that are able to achieve a higher mean
score than any other known opponent. All of the trained strategies use many
rounds of the history of play (some using all of it) and outperform memory-one
strategies.

Directly designing an effective strategy is no easy task. Of all the actively
learning strategies in the library, only DBS performs well. The authors trained
a neural network to predict opponents' next moves with high accuracy but basing
a strategy on the predictions proved difficult -- it is necessary to consider
the downstream consequences of a defection and direct training of strategy
outputs generally performed better.

These results suggest that, in opposition to historical tournament results
and community folklore, that complex strategies can be very effective for the
IPD. It is not the complexity of strategies that is disadvantageous, rather it
is apparently too difficult for humans to design complex yet effective
strategies, DBS being a notable exception.

Furthermore, dealing with noise is difficult for most strategies. Two strategies
designed specifically to account for noise, DBS and OmegaTFT, perform well and
only DBS performs better than directly trained finite state machine strategies.

Of all the strategies trained to maximize their mean score all are generally
cooperative, not defecting until the opponent defects. Maximizing for individual
performance across a collection of opponents leads to mutual cooperation despite
the fact that mutual cooperation is an unstable equilibrium for the prisoner's
dilemma. Specifically we note that the reinforcement learning process does not
lead to exploitative zero determinant strategies, which may be a result of the
collection of training strategies, many of which retailate harshly.

Finally, we note that as the library grows, the top performing strategies
sometimes shuffle, and are not retrained regularly. The precise parameters
that are optimal will depend on the pool of opponents. Moreover we have not
extensively trained strategies to determine the minimum parameters that are
sufficient -- neural networks with fewer nodes and features and finite state
machines with fewer states may suffice.
% TODO: Cite Ashlock paper on parameter capacity for trained strategies


Future work:
* spatial tournaments and other variants
* Additional strategy archetypes by the Ashlocks, e.g. function stacks, binary
decision players
* further refine features and training parameters


\section*{Acknowledgements}

This work was performed using the computational facilities of the Advanced
Research Computing @ Cardiff (ARCCA) Division, Cardiff University.

\printbibliography
